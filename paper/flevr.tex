% Template for PLoS
% Version 3.5 March 2018
%
% % % % % % % % % % % % % % % % % % % % % %
%
% -- IMPORTANT NOTE
%
% This template contains comments intended
% to minimize problems and delays during our production
% process. Please follow the template instructions
% whenever possible.
%
% % % % % % % % % % % % % % % % % % % % % % %
%
% Once your paper is accepted for publication,
% PLEASE REMOVE ALL TRACKED CHANGES in this file
% and leave only the final text of your manuscript.
% PLOS recommends the use of latexdiff to track changes during review, as this will help to maintain a clean tex file.
% Visit https://www.ctan.org/pkg/latexdiff?lang=en for info or contact us at latex@plos.org.
%
%
% There are no restrictions on package use within the LaTeX files except that
% no packages listed in the template may be deleted.
%
% Please do not include colors or graphics in the text.
%
% The manuscript LaTeX source should be contained within a single file (do not use \input, \externaldocument, or similar commands).
%
% % % % % % % % % % % % % % % % % % % % % % %
%
% -- FIGURES AND TABLES
%
% Please include tables/figure captions directly after the paragraph where they are first cited in the text.
%
% DO NOT INCLUDE GRAPHICS IN YOUR MANUSCRIPT
% - Figures should be uploaded separately from your manuscript file.
% - Figures generated using LaTeX should be extracted and removed from the PDF before submission.
% - Figures containing multiple panels/subfigures must be combined into one image file before submission.
% For figure citations, please use "Fig" instead of "Figure".
% See http://journals.plos.org/plosone/s/figures for PLOS figure guidelines.
%
% Tables should be cell-based and may not contain:
% - spacing/line breaks within cells to alter layout or alignment
% - do not nest tabular environments (no tabular environments within tabular environments)
% - no graphics or colored text (cell background color/shading OK)
% See http://journals.plos.org/plosone/s/tables for table guidelines.
%
% For tables that exceed the width of the text column, use the adjustwidth environment as illustrated in the example table in text below.
%
% % % % % % % % % % % % % % % % % % % % % % % %
%
% -- EQUATIONS, MATH SYMBOLS, SUBSCRIPTS, AND SUPERSCRIPTS
%
% IMPORTANT
% Below are a few tips to help format your equations and other special characters according to our specifications. For more tips to help reduce the possibility of formatting errors during conversion, please see our LaTeX guidelines at http://journals.plos.org/plosone/s/latex
%
% For inline equations, please be sure to include all portions of an equation in the math environment.
%
% Do not include text that is not math in the math environment.
%
% Please add line breaks to long display equations when possible in order to fit size of the column.
%
% For inline equations, please do not include punctuation (commas, etc) within the math environment unless this is part of the equation.
%
% When adding superscript or subscripts outside of brackets/braces, please group using {}.
%
% Do not use \cal for caligraphic font.  Instead, use \mathcal{}
%
% % % % % % % % % % % % % % % % % % % % % % % %
%
% Please contact latex@plos.org with any questions.
%
% % % % % % % % % % % % % % % % % % % % % % % %

\documentclass[10pt,letterpaper]{article}
\usepackage[top=0.85in,left=2.75in,footskip=0.75in]{geometry}

% amsmath and amssymb packages, useful for mathematical formulas and symbols
\usepackage{amsmath,amssymb}

% Use adjustwidth environment to exceed column width (see example table in text)
\usepackage{changepage}


% textcomp package and marvosym package for additional characters
\usepackage{textcomp,marvosym}

% cite package, to clean up citations in the main text. Do not remove.
% \usepackage{cite}

% Use nameref to cite supporting information files (see Supporting Information section for more info)
\usepackage{nameref,hyperref}

% line numbers
\usepackage[right]{lineno}

% ligatures disabled
\usepackage{microtype}
\DisableLigatures[f]{encoding = *, family = * }

% color can be used to apply background shading to table cells only
\usepackage[table]{xcolor}

% array package and thick rules for tables
\usepackage{array}

% create "+" rule type for thick vertical lines
\newcolumntype{+}{!{\vrule width 2pt}}

% create \thickcline for thick horizontal lines of variable length
\newlength\savedwidth
\newcommand\thickcline[1]{%
  \noalign{\global\savedwidth\arrayrulewidth\global\arrayrulewidth 2pt}%
  \cline{#1}%
  \noalign{\vskip\arrayrulewidth}%
  \noalign{\global\arrayrulewidth\savedwidth}%
}

% \thickhline command for thick horizontal lines that span the table
\newcommand\thickhline{\noalign{\global\savedwidth\arrayrulewidth\global\arrayrulewidth 2pt}%
\hline
\noalign{\global\arrayrulewidth\savedwidth}}


% Remove comment for double spacing
%\usepackage{setspace}
%\doublespacing

% Text layout
\raggedright
\setlength{\parindent}{0.5cm}
\textwidth 5.25in
\textheight 8.75in

% Bold the 'Figure #' in the caption and separate it from the title/caption with a period
% Captions will be left justified
\usepackage[aboveskip=1pt,labelfont=bf,labelsep=period,justification=raggedright,singlelinecheck=off]{caption}
\renewcommand{\figurename}{Fig}

% Use the PLoS provided BiBTeX style
% \bibliographystyle{plos2015}

% Remove brackets from numbering in List of References
\makeatletter
\renewcommand{\@biblabel}[1]{\quad#1.}
\makeatother



% Header and Footer with logo
\usepackage{lastpage,fancyhdr,graphicx}
\usepackage{epstopdf}
%\pagestyle{myheadings}
\pagestyle{fancy}
\fancyhf{}
%\setlength{\headheight}{27.023pt}
%\lhead{\includegraphics[width=2.0in]{PLOS-submission.eps}}
\rfoot{\thepage/\pageref{LastPage}}
\renewcommand{\headrulewidth}{0pt}
\renewcommand{\footrule}{\hrule height 2pt \vspace{2mm}}
\fancyheadoffset[L]{2.25in}
\fancyfootoffset[L]{2.25in}
\lfoot{\today}

%% Include all macros below

\newcommand{\lorem}{{\bf LOREM}}
\newcommand{\ipsum}{{\bf IPSUM}}

% Pandoc syntax highlighting
\usepackage{color}
\usepackage{fancyvrb}
\newcommand{\VerbBar}{|}
\newcommand{\VERB}{\Verb[commandchars=\\\{\}]}
\DefineVerbatimEnvironment{Highlighting}{Verbatim}{commandchars=\\\{\}}
% Add ',fontsize=\small' for more characters per line
\usepackage{framed}
\definecolor{shadecolor}{RGB}{248,248,248}
\newenvironment{Shaded}{\begin{snugshade}}{\end{snugshade}}
\newcommand{\AlertTok}[1]{\textcolor[rgb]{0.94,0.16,0.16}{#1}}
\newcommand{\AnnotationTok}[1]{\textcolor[rgb]{0.56,0.35,0.01}{\textbf{\textit{#1}}}}
\newcommand{\AttributeTok}[1]{\textcolor[rgb]{0.13,0.29,0.53}{#1}}
\newcommand{\BaseNTok}[1]{\textcolor[rgb]{0.00,0.00,0.81}{#1}}
\newcommand{\BuiltInTok}[1]{#1}
\newcommand{\CharTok}[1]{\textcolor[rgb]{0.31,0.60,0.02}{#1}}
\newcommand{\CommentTok}[1]{\textcolor[rgb]{0.56,0.35,0.01}{\textit{#1}}}
\newcommand{\CommentVarTok}[1]{\textcolor[rgb]{0.56,0.35,0.01}{\textbf{\textit{#1}}}}
\newcommand{\ConstantTok}[1]{\textcolor[rgb]{0.56,0.35,0.01}{#1}}
\newcommand{\ControlFlowTok}[1]{\textcolor[rgb]{0.13,0.29,0.53}{\textbf{#1}}}
\newcommand{\DataTypeTok}[1]{\textcolor[rgb]{0.13,0.29,0.53}{#1}}
\newcommand{\DecValTok}[1]{\textcolor[rgb]{0.00,0.00,0.81}{#1}}
\newcommand{\DocumentationTok}[1]{\textcolor[rgb]{0.56,0.35,0.01}{\textbf{\textit{#1}}}}
\newcommand{\ErrorTok}[1]{\textcolor[rgb]{0.64,0.00,0.00}{\textbf{#1}}}
\newcommand{\ExtensionTok}[1]{#1}
\newcommand{\FloatTok}[1]{\textcolor[rgb]{0.00,0.00,0.81}{#1}}
\newcommand{\FunctionTok}[1]{\textcolor[rgb]{0.13,0.29,0.53}{\textbf{#1}}}
\newcommand{\ImportTok}[1]{#1}
\newcommand{\InformationTok}[1]{\textcolor[rgb]{0.56,0.35,0.01}{\textbf{\textit{#1}}}}
\newcommand{\KeywordTok}[1]{\textcolor[rgb]{0.13,0.29,0.53}{\textbf{#1}}}
\newcommand{\NormalTok}[1]{#1}
\newcommand{\OperatorTok}[1]{\textcolor[rgb]{0.81,0.36,0.00}{\textbf{#1}}}
\newcommand{\OtherTok}[1]{\textcolor[rgb]{0.56,0.35,0.01}{#1}}
\newcommand{\PreprocessorTok}[1]{\textcolor[rgb]{0.56,0.35,0.01}{\textit{#1}}}
\newcommand{\RegionMarkerTok}[1]{#1}
\newcommand{\SpecialCharTok}[1]{\textcolor[rgb]{0.81,0.36,0.00}{\textbf{#1}}}
\newcommand{\SpecialStringTok}[1]{\textcolor[rgb]{0.31,0.60,0.02}{#1}}
\newcommand{\StringTok}[1]{\textcolor[rgb]{0.31,0.60,0.02}{#1}}
\newcommand{\VariableTok}[1]{\textcolor[rgb]{0.00,0.00,0.00}{#1}}
\newcommand{\VerbatimStringTok}[1]{\textcolor[rgb]{0.31,0.60,0.02}{#1}}
\newcommand{\WarningTok}[1]{\textcolor[rgb]{0.56,0.35,0.01}{\textbf{\textit{#1}}}}

% tightlist command for lists without linebreak
\providecommand{\tightlist}{%
  \setlength{\itemsep}{0pt}\setlength{\parskip}{0pt}}


% Pandoc citation processing
\newlength{\cslhangindent}
\setlength{\cslhangindent}{1.5em}
\newlength{\csllabelwidth}
\setlength{\csllabelwidth}{3em}
\newlength{\cslentryspacingunit} % times entry-spacing
\setlength{\cslentryspacingunit}{\parskip}
% for Pandoc 2.8 to 2.10.1
\newenvironment{cslreferences}%
  {}%
  {\par}
% For Pandoc 2.11+
\newenvironment{CSLReferences}[2] % #1 hanging-ident, #2 entry spacing
 {% don't indent paragraphs
  \setlength{\parindent}{0pt}
  % turn on hanging indent if param 1 is 1
  \ifodd #1
  \let\oldpar\par
  \def\par{\hangindent=\cslhangindent\oldpar}
  \fi
  % set entry spacing
  \setlength{\parskip}{#2\cslentryspacingunit}
 }%
 {}
\usepackage{calc}
\newcommand{\CSLBlock}[1]{#1\hfill\break}
\newcommand{\CSLLeftMargin}[1]{\parbox[t]{\csllabelwidth}{#1}}
\newcommand{\CSLRightInline}[1]{\parbox[t]{\linewidth - \csllabelwidth}{#1}\break}
\newcommand{\CSLIndent}[1]{\hspace{\cslhangindent}#1}



\usepackage{forarray}
\usepackage{xstring}
\newcommand{\getIndex}[2]{
  \ForEach{,}{\IfEq{#1}{\thislevelitem}{\number\thislevelcount\ExitForEach}{}}{#2}
}

\setcounter{secnumdepth}{0}

\newcommand{\getAff}[1]{
  \getIndex{#1}{KPWHRI,Fred Hutch,UW}
}

\begin{document}
\vspace*{0.2in}


% Title must be 250 characters or less.
\begin{flushleft}
{\Large
\textbf\newline{flevr: Flexible, Ensemble-based Variable Selection in
R} % Please use "sentence case" for title and headings (capitalize only the first word in a title (or heading), the first word in a subtitle (or subheading), and any proper nouns).
}
\newline
% Insert author names, affiliations and corresponding author email (do not include titles, positions, or degrees).
\\
Brian D. Williamson\textsuperscript{\getAff{KPWHRI}, \getAff{Fred
Hutch}, \getAff{UW}}\textsuperscript{*},
Ying Huang\textsuperscript{\getAff{Fred Hutch}, \getAff{UW}}\\
\bigskip
\textbf{\getAff{KPWHRI}}Biostatistics Division, Kaiser Permanente
Washington Health Research Institute, Seattle, WA\\
\textbf{\getAff{Fred Hutch}}Vaccine and Infectious Disease Division,
Fred Hutchinson Cancer Center, Seattle, WA\\
\textbf{\getAff{UW}}Department of Biostatistics, University of
Washington, Seattle, WA\\
\bigskip
* Corresponding author: brian.d.williamson@kp.org\\
\end{flushleft}
% Please keep the abstract below 300 words
\section*{Abstract}
Variable selection in missing-data contexts is complicated by two
issues: (i) a desire for the results to be robust to departures from
simplifying modelling assumptions, and (ii) the need to incorporate a
missing-data approach. Often, researchers turn to machine learning to
increase robustness, but the options for missing-data variable selection
when machine learning tools have been used are neither easily
interpretable nor implemented in easy-to-use software. The flevr R
package provides an open-source tool for performing variable selection
using flexible, ensemble machine learning algorithms. An intrinsic
selection approach based on population variable importance solves these
issues. We illustrate the use of the package with several examples.

% Please keep the Author Summary between 150 and 200 words
% Use first person. PLOS ONE authors please skip this step.
% Author Summary not valid for PLOS ONE submissions.

\linenumbers

% Use "Eq" instead of "Equation" for equation citations.
\hypertarget{introduction}{%
\section{Introduction}\label{introduction}}

Variable selection is a common goal in biomedical research, among other
fields. Traditional tools for variable selection are generally not
robust to departures from possibly restrictive modelling assumptions
{[}1{]}. Often, researchers turn to machine learning to increase
robustness, but the options for missing-data variable selection when
machine learning tools have been used are not easily interpretable.
However, in settings with missing data, procedures to combine variable
selection results across imputed datasets become more difficult to
interpret when machine learning tools are used {[}2{]}.

Often, in missing-data settings, variable selection is accomplished in
one of two ways. Methods for variable selection with full observation of
all variables can be adapted to the missing-data setting by using
likelihood-based methods {[}3{]} or methods based on inverse probability
weighting {[}4--7{]}. While these approaches can be quite useful, they
are often tailored to specific data-generating or missing-data
processes, rendering their implementation in software difficult. In
contrast, approaches based on multiple imputation
{[}8,9{]}---implemented in the R package \texttt{mice} {[}10{]}, among
other packages---are easy to incorporate into analysis pipelines and
therefore are widely used. After the missing data are multiply imputed,
any variable selection method can be used on the imputed datasets.
However, the selected variables across different imputed datasets may be
different, and thus must be harmonized across all imputed datasets;
often, variables that are selected in some large proportion of datasets
are retained in the final selected set {[}2{]}.

There is a substantial literature on variable selection with
fully-observed data; software implementations are also common, with 95
packages on the Comprehensive R Archive Network (CRAN) having ``variable
selection'' in the title as of February 2024. Several commonly-used
methods include the lasso {[}11{]}, implemented in the R package
\texttt{glmnet} {[}12{]}; stability selection {[}13{]}, implemented in
the R package \texttt{stabs} {[}{[}14{]};Hofner:pkg\_stabs:2021{]}; and
knockoffs {[}15{]}, implemented in the R package \texttt{knockoff}
{[}16{]}. However, prior to our package \texttt{flevr} (described
below), only one R package available on CRAN, \texttt{miselect}
{[}17{]}, explicitly mentions missing data in its title. This package
forces selection of the same variables across multiply-imputed datasets
{[}18{]}; while potentially useful in some contexts, this may be
undesirable in other contexts. Additionally, using a single algorithm to
perform variable selection carries the risk that if the underlying model
assumptions are not met (e.g., a linear regression model for the lasso),
the selected variables might not be correct. Approaches based on
ensembles of multiple algorithms may be more robust to this
misspecification; examples include random forests {[}19{]}, majority
vote over the ensemble {[}20{]}, and rank-based weighting {[}21{]}, but
there is no \texttt{R} implementation of these approaches on CRAN.

The \texttt{flevr} package for \texttt{R} {[}22{]} provides researchers
with the tools necessary to perform variable selection in settings with
missing data based on flexible prediction approaches. One approach that
we implement is to use the Super Learner {[}23{]} to perform variable
selection based on an ensemble of individual prediction algorithms, an
approach we call \emph{extrinsic selection}. The Super Learner has
asymptotic and finite-sample guarantees on its prediction performance
{[}23{]}, improving robustness of selected variables to model
misspecification. The second approach is to base variable selection on
\emph{intrinsic variable importance} {[}24,25{]}, which quantifies the
population prediction potential of features. The package facilitates
both handling missing data using multiple imputation {[}8{]} and
incorporating state-of-the-art machine learning algorithms to perform
variable selection. The \emph{intrinsic selection} approach
(\texttt{intrinsic\_selection}) is based on population variable
importance {[}26{]}; this approach outputs a single set of selected
variables, in contrast to other procedures that require post-hoc
harmonization of multiple selected sets {[}2{]}. To our knowledge, this
is the first \texttt{R} package implementing both general-purpose
ensemble-based variable selection and intrinsic variable selection.

\hypertarget{methods}{%
\section{Methods}\label{methods}}

This section should involve some math, but not be as deep as either the
original intrinsic selection paper or the companion paper

\hypertarget{motivating-dataset}{%
\subsection{Motivating dataset}\label{motivating-dataset}}

describe the dataset

\hypertarget{extrinsic-variable-selection}{%
\subsection{Extrinsic variable
selection}\label{extrinsic-variable-selection}}

cite the companion paper; add math supporting this method

\hypertarget{intrinsic-variable-selection}{%
\subsection{Intrinsic variable
selection}\label{intrinsic-variable-selection}}

{[}26{]}; add math supporting this method

\hypertarget{examples}{%
\section{Examples}\label{examples}}

This section should serve as a guide to using the \texttt{flevr} package
--- we will cover the main functions for doing extrinsic and intrinsic
variable selection using a simulated data example.

First, we create some data:

\begin{Shaded}
\begin{Highlighting}[]
\CommentTok{\# generate the data {-}{-} note that this is a simple setting, for speed}
\FunctionTok{set.seed}\NormalTok{(}\DecValTok{4747}\NormalTok{)}
\NormalTok{p }\OtherTok{\textless{}{-}} \DecValTok{2}
\NormalTok{n }\OtherTok{\textless{}{-}} \DecValTok{500}
\CommentTok{\# generate features}
\NormalTok{x }\OtherTok{\textless{}{-}} \FunctionTok{replicate}\NormalTok{(p, stats}\SpecialCharTok{::}\FunctionTok{rnorm}\NormalTok{(n, }\DecValTok{0}\NormalTok{, }\DecValTok{1}\NormalTok{))}
\NormalTok{x\_df }\OtherTok{\textless{}{-}} \FunctionTok{as.data.frame}\NormalTok{(x)}
\NormalTok{x\_names }\OtherTok{\textless{}{-}} \FunctionTok{names}\NormalTok{(x\_df)}
\CommentTok{\# generate outcomes}
\NormalTok{y }\OtherTok{\textless{}{-}} \DecValTok{1} \SpecialCharTok{+} \FloatTok{0.5} \SpecialCharTok{*}\NormalTok{ x[, }\DecValTok{1}\NormalTok{] }\SpecialCharTok{+} \FloatTok{0.75} \SpecialCharTok{*}\NormalTok{ x[, }\DecValTok{2}\NormalTok{] }\SpecialCharTok{+}\NormalTok{ stats}\SpecialCharTok{::}\FunctionTok{rnorm}\NormalTok{(n, }\DecValTok{0}\NormalTok{, }\DecValTok{1}\NormalTok{)}
\end{Highlighting}
\end{Shaded}

This creates a matrix of covariates \texttt{x} with 2 columns and a
vector \texttt{y} of normally-distributed outcome values for a sample of
\texttt{n\ =\ 500} study participants.

There are two main types of variable selection available in
\texttt{flevr}: extrinsic and intrinsic. Extrinsic selection is the most
common type of variable selection: in this approach, a given algorithm
(and perhaps its associated algorithm-specific variable importance) is
used for variable selection. The lasso is a widely-used example of
extrinsic selection. Intrinsic selection, on the other hand, uses
estimated intrinsic variable importance (a population quantity) to
perform variable selection. This intrinsic importance is both defined
and estimated in a model-agnostic manner.

\hypertarget{extrinsic-variable-selection-1}{%
\subsection{Extrinsic variable
selection}\label{extrinsic-variable-selection-1}}

We recommend using the Super Learner {[}23{]} to do extrinsic variable
selection to protect against model misspecification. This requires
specifying a \emph{library} of \emph{candidate learners} (e.g., lasso,
random forests). We can do this in \texttt{flevr} using the following
code:

\begin{Shaded}
\begin{Highlighting}[]
\FunctionTok{library}\NormalTok{(}\StringTok{"flevr"}\NormalTok{)}
\FunctionTok{set.seed}\NormalTok{(}\DecValTok{1234}\NormalTok{)}
\CommentTok{\# fit a Super Learner ensemble; note its simplicity, for speed}
\FunctionTok{library}\NormalTok{(}\StringTok{"SuperLearner"}\NormalTok{)}
\NormalTok{learners }\OtherTok{\textless{}{-}} \FunctionTok{c}\NormalTok{(}\StringTok{"SL.glm"}\NormalTok{, }\StringTok{"SL.mean"}\NormalTok{)}
\NormalTok{V }\OtherTok{\textless{}{-}} \DecValTok{2}
\NormalTok{fit }\OtherTok{\textless{}{-}}\NormalTok{ SuperLearner}\SpecialCharTok{::}\FunctionTok{SuperLearner}\NormalTok{(}\AttributeTok{Y =}\NormalTok{ y, }\AttributeTok{X =}\NormalTok{ x\_df,}
                                  \AttributeTok{SL.library =}\NormalTok{ learners,}
                                  \AttributeTok{cvControl =} \FunctionTok{list}\NormalTok{(}\AttributeTok{V =}\NormalTok{ V))}
\CommentTok{\# extract importance based on the whole Super Learner}
\NormalTok{sl\_importance\_all }\OtherTok{\textless{}{-}} \FunctionTok{extract\_importance\_SL}\NormalTok{(}
  \AttributeTok{fit =}\NormalTok{ fit, }\AttributeTok{feature\_names =}\NormalTok{ x\_names, }\AttributeTok{import\_type =} \StringTok{"all"}
\NormalTok{)}
\NormalTok{sl\_importance\_all}

\CommentTok{\# A tibble: 2 × 2}
\CommentTok{\#   feature  rank}
\CommentTok{\#   \textless{}chr\textgreater{}   \textless{}dbl\textgreater{}}
\CommentTok{\# 1 V2       1.01}
\CommentTok{\# 2 V1       1.99}
\end{Highlighting}
\end{Shaded}

These results suggest that feature 2 is more important than feature 1
within the Super Learner ensemble (since a lower rank is better). If we
want to scrutinize the importance of features within the best-fitting
algorithm in the Super Learner ensemble, we can do the following:

\begin{Shaded}
\begin{Highlighting}[]
\CommentTok{\# check the best{-}fitting learner}
\NormalTok{fit}
\CommentTok{\# Call:  SuperLearner::SuperLearner(Y = y, X = x\_df, SL.library = learners, cvControl = list(V = V)) }
 
 
 
\CommentTok{\#                  Risk       Coef}
\CommentTok{\# SL.glm\_All  0.9614026 0.98842732}
\CommentTok{\# SL.mean\_All 1.8183525 0.01157268}

\CommentTok{\# importance within the best{-}fitting learner}
\NormalTok{sl\_importance\_best }\OtherTok{\textless{}{-}} \FunctionTok{extract\_importance\_SL}\NormalTok{(}
  \AttributeTok{fit =}\NormalTok{ fit, }\AttributeTok{feature\_names =}\NormalTok{ x\_names, }\AttributeTok{import\_type =} \StringTok{"best"}
\NormalTok{)}
\NormalTok{sl\_importance\_best}
\CommentTok{\# \# A tibble: 2 × 2}
\CommentTok{\#   feature  rank}
\CommentTok{\#   \textless{}chr\textgreater{}   \textless{}int\textgreater{}}
\CommentTok{\# 1 V2          1}
\CommentTok{\# 2 V1          2}
\end{Highlighting}
\end{Shaded}

Finally, to do variable selection, we need to select a threshold for the
variable importance rank (ideally before looking at the data). In this
case, since there are only two variables, we choose a threshold of 1.5,
which means we will select only one variable:

\begin{Shaded}
\begin{Highlighting}[]
\NormalTok{extrinsic\_selected }\OtherTok{\textless{}{-}} \FunctionTok{extrinsic\_selection}\NormalTok{(}
  \AttributeTok{fit =}\NormalTok{ fit, }\AttributeTok{feature\_names =}\NormalTok{ x\_names, }\AttributeTok{threshold =} \FloatTok{1.5}\NormalTok{, }\AttributeTok{import\_type =} \StringTok{"all"}
\NormalTok{)}
\NormalTok{extrinsic\_selected}
\CommentTok{\# A tibble: 2 × 3}
\CommentTok{\#   feature  rank selected}
\CommentTok{\#   \textless{}chr\textgreater{}   \textless{}dbl\textgreater{} \textless{}lgl\textgreater{}   }
\CommentTok{\# 1 V2       1.01 TRUE    }
\CommentTok{\# 2 V1       1.99 FALSE  }
\end{Highlighting}
\end{Shaded}

In this case, we select only variable 2.

\hypertarget{intrinsic-variable-selection-1}{%
\subsection{Intrinsic variable
selection}\label{intrinsic-variable-selection-1}}

Intrinsic variable selection is based on population variable importance
{[}24,25{]}. Intrinsic selection {[}26{]} also uses the Super Learner
under the hood, and requires specifying a useful \emph{measure of
predictiveness} (e.g., R-squared or classification accuracy). The first
step in doing intrinsic selection is estimating the variable importance:

\begin{Shaded}
\begin{Highlighting}[]
\FunctionTok{set.seed}\NormalTok{(}\DecValTok{1234}\NormalTok{)}

\CommentTok{\# set up a library for SuperLearner}
\NormalTok{learners }\OtherTok{\textless{}{-}} \StringTok{"SL.glm"}
\NormalTok{univariate\_learners }\OtherTok{\textless{}{-}} \StringTok{"SL.glm"}
\NormalTok{V }\OtherTok{\textless{}{-}} \DecValTok{2}

\CommentTok{\# estimate the SPVIMs}
\FunctionTok{library}\NormalTok{(}\StringTok{"vimp"}\NormalTok{)}
\NormalTok{est }\OtherTok{\textless{}{-}} \FunctionTok{suppressWarnings}\NormalTok{(}
  \FunctionTok{sp\_vim}\NormalTok{(}\AttributeTok{Y =}\NormalTok{ y, }\AttributeTok{X =}\NormalTok{ x, }\AttributeTok{V =}\NormalTok{ V, }\AttributeTok{type =} \StringTok{"r\_squared"}\NormalTok{,}
              \AttributeTok{SL.library =}\NormalTok{ learners, }\AttributeTok{gamma =}\NormalTok{ .}\DecValTok{1}\NormalTok{, }\AttributeTok{alpha =} \FloatTok{0.05}\NormalTok{, }\AttributeTok{delta =} \DecValTok{0}\NormalTok{,}
              \AttributeTok{cvControl =} \FunctionTok{list}\NormalTok{(}\AttributeTok{V =}\NormalTok{ V), }\AttributeTok{env =} \FunctionTok{environment}\NormalTok{())}
\NormalTok{)}
\NormalTok{est}
\CommentTok{\# Variable importance estimates:}
\CommentTok{\#       Estimate  SE         95\% CI                  VIMP \textgreater{} 0 p{-}value     }
\CommentTok{\# s = 1 0.1515809 0.06090463 [0.03221005, 0.2709518] TRUE     1.330062e{-}03}
\CommentTok{\# s = 2 0.2990449 0.06565597 [0.17036157, 0.4277282] TRUE     6.863052e{-}09}
\end{Highlighting}
\end{Shaded}

This procedure again shows (correctly) that variable 2 is more important
than variable 1 in this population, since the point estimate of variable
importance is larger.

The next step is to choose an error rate to control and a method for
controlling the family-wise error rate. Here, we choose to control the
generalized family-wise error rate, setting \texttt{k\ =\ 1} and
\texttt{alpha\ =\ 0.2} (so controlling the probability of making more
than 1 type I error at 20\%), and choose Holm-adjusted p-values to
control the initial family-wise error rate:

\begin{Shaded}
\begin{Highlighting}[]
\NormalTok{intrinsic\_set }\OtherTok{\textless{}{-}} \FunctionTok{intrinsic\_selection}\NormalTok{(}
  \AttributeTok{spvim\_ests =}\NormalTok{ est, }\AttributeTok{sample\_size =}\NormalTok{ n, }\AttributeTok{alpha =} \FloatTok{0.2}\NormalTok{, }\AttributeTok{feature\_names =}\NormalTok{ x\_names,}
  \AttributeTok{control =} \FunctionTok{list}\NormalTok{( }\AttributeTok{quantity =} \StringTok{"gFWER"}\NormalTok{, }\AttributeTok{base\_method =} \StringTok{"Holm"}\NormalTok{, }\AttributeTok{k =} \DecValTok{1}\NormalTok{)}
\NormalTok{)}
\NormalTok{intrinsic\_set}
\CommentTok{\# A tibble: 2 × 6}
\CommentTok{\#   feature   est       p\_value adjusted\_p\_value  rank selected}
\CommentTok{\#   \textless{}chr\textgreater{}   \textless{}dbl\textgreater{}         \textless{}dbl\textgreater{}            \textless{}dbl\textgreater{} \textless{}dbl\textgreater{} \textless{}lgl\textgreater{}   }
\CommentTok{\# 1 V1      0.152 0.00133           0.00133          2 TRUE    }
\CommentTok{\# 2 V2      0.299 0.00000000686     0.0000000137     1 TRUE}
\end{Highlighting}
\end{Shaded}

In this case, we select both variables.

\hypertarget{variable-selection-with-missing-data}{%
\subsection{Variable selection with missing
data}\label{variable-selection-with-missing-data}}

Settings with missing data render variable selection more challenging.
In this section, we consider a simulated dataset inspired by data
collected by the Early Detection Research Network. Biomarkers developed
at six ``labs'' are validated at at least one of four ``validation
sites'' on 306 cysts. The data also include two binary outcome
variables: whether or not the cyst was classified as mucinous, and
whether or not the cyst was determined to have high malignant potential.
The data contain some missing information, which complicates variable
selection; only 212 cysts have complete information. We will use AUC to
measure intrinsic importance. We begin by loading the data and
performing multiple imputation.

\begin{Shaded}
\begin{Highlighting}[]
\CommentTok{\# load the dataset}
\FunctionTok{data}\NormalTok{(}\StringTok{"biomarkers"}\NormalTok{)}
\FunctionTok{library}\NormalTok{(}\StringTok{"dplyr"}\NormalTok{)}
\CommentTok{\# set up vector "y" of outcomes and matrix "x" of features}
\NormalTok{y }\OtherTok{\textless{}{-}}\NormalTok{ biomarkers}\SpecialCharTok{$}\NormalTok{mucinous}
\NormalTok{x }\OtherTok{\textless{}{-}}\NormalTok{ biomarkers }\SpecialCharTok{\%\textgreater{}\%}
  \FunctionTok{na.omit}\NormalTok{() }\SpecialCharTok{\%\textgreater{}\%}
  \FunctionTok{select}\NormalTok{(}\FunctionTok{starts\_with}\NormalTok{(}\StringTok{"lab"}\NormalTok{), }\FunctionTok{starts\_with}\NormalTok{(}\StringTok{"cea"}\NormalTok{))}
\NormalTok{x\_names }\OtherTok{\textless{}{-}} \FunctionTok{names}\NormalTok{(x)}
\FunctionTok{library}\NormalTok{(}\StringTok{"mice"}\NormalTok{)}
\FunctionTok{set.seed}\NormalTok{(}\DecValTok{20231121}\NormalTok{)}
\NormalTok{mi\_biomarkers }\OtherTok{\textless{}{-}}\NormalTok{ mice}\SpecialCharTok{::}\FunctionTok{mice}\NormalTok{(}\AttributeTok{data =}\NormalTok{ biomarkers, }\AttributeTok{m =} \DecValTok{5}\NormalTok{, }\AttributeTok{printFlag =} \ConstantTok{FALSE}\NormalTok{)}
\NormalTok{imputed\_biomarkers }\OtherTok{\textless{}{-}}\NormalTok{ mice}\SpecialCharTok{::}\FunctionTok{complete}\NormalTok{(mi\_biomarkers, }\AttributeTok{action =} \StringTok{"long"}\NormalTok{) }\SpecialCharTok{\%\textgreater{}\%}
  \FunctionTok{rename}\NormalTok{(}\AttributeTok{imp =}\NormalTok{ .imp, }\AttributeTok{id =}\NormalTok{ .id)}
\end{Highlighting}
\end{Shaded}

\hypertarget{extrinsic-selection-with-missing-data}{%
\subsubsection{Extrinsic selection with missing
data}\label{extrinsic-selection-with-missing-data}}

We can perform extrinsic variable selection using the imputed data.
First, we fit a Super Learner and perform extrinsic variable selection
for each imputed dataset. Then, we select a final set of variables based
on those that are selected in a pre-specified number of imputed datasets
(e.g., 3 of 5) {[}2{]}. Again, we use a rank of 5 for each imputed
dataset to select variables.

\begin{Shaded}
\begin{Highlighting}[]
\NormalTok{extrinsic\_learners }\OtherTok{\textless{}{-}} \FunctionTok{c}\NormalTok{(}\StringTok{"SL.glm"}\NormalTok{, }\StringTok{"SL.ranger.imp"}\NormalTok{, }\StringTok{"SL.xgboost"}\NormalTok{)}
\FunctionTok{set.seed}\NormalTok{(}\DecValTok{20231121}\NormalTok{)}
\CommentTok{\# set up a list to collect selected sets}
\NormalTok{all\_selected\_vars }\OtherTok{\textless{}{-}} \FunctionTok{vector}\NormalTok{(}\StringTok{"list"}\NormalTok{, }\AttributeTok{length =} \DecValTok{5}\NormalTok{)}
\CommentTok{\# for each imputed dataset, do extrinsic selection}
\ControlFlowTok{for}\NormalTok{ (i }\ControlFlowTok{in} \DecValTok{1}\SpecialCharTok{:}\DecValTok{5}\NormalTok{) \{}
  \CommentTok{\# fit a Super Learner}
\NormalTok{  these\_data }\OtherTok{\textless{}{-}}\NormalTok{ imputed\_biomarkers }\SpecialCharTok{\%\textgreater{}\%}
    \FunctionTok{filter}\NormalTok{(imp }\SpecialCharTok{==}\NormalTok{ i)}
\NormalTok{  this\_y }\OtherTok{\textless{}{-}}\NormalTok{ these\_data}\SpecialCharTok{$}\NormalTok{mucinous}
\NormalTok{  this\_x }\OtherTok{\textless{}{-}}\NormalTok{ these\_data }\SpecialCharTok{\%\textgreater{}\%}
    \FunctionTok{select}\NormalTok{(}\FunctionTok{starts\_with}\NormalTok{(}\StringTok{"lab"}\NormalTok{), }\FunctionTok{starts\_with}\NormalTok{(}\StringTok{"cea"}\NormalTok{))}
\NormalTok{  this\_x\_df }\OtherTok{\textless{}{-}} \FunctionTok{as.data.frame}\NormalTok{(this\_x)}
\NormalTok{  fit }\OtherTok{\textless{}{-}}\NormalTok{ SuperLearner}\SpecialCharTok{::}\FunctionTok{SuperLearner}\NormalTok{(}\AttributeTok{Y =}\NormalTok{ this\_y, }\AttributeTok{X =}\NormalTok{ this\_x\_df,}
                                  \AttributeTok{SL.library =}\NormalTok{ extrinsic\_learners,}
                                  \AttributeTok{cvControl =} \FunctionTok{list}\NormalTok{(}\AttributeTok{V =}\NormalTok{ V),}
                                  \AttributeTok{family =} \StringTok{"binomial"}\NormalTok{)}
  \CommentTok{\# do extrinsic selection}
\NormalTok{  all\_selected\_vars[[i]] }\OtherTok{\textless{}{-}} \FunctionTok{extrinsic\_selection}\NormalTok{(}
    \AttributeTok{fit =}\NormalTok{ fit, }\AttributeTok{feature\_names =}\NormalTok{ x\_names, }\AttributeTok{threshold =} \DecValTok{5}\NormalTok{, }\AttributeTok{import\_type =} \StringTok{"all"}
\NormalTok{  )}\SpecialCharTok{$}\NormalTok{selected}
\NormalTok{\}}
\CommentTok{\# perform extrinsic variable selection}
\NormalTok{selected\_vars }\OtherTok{\textless{}{-}} \FunctionTok{pool\_selected\_sets}\NormalTok{(}\AttributeTok{sets =}\NormalTok{ all\_selected\_vars, }\AttributeTok{threshold =} \DecValTok{3} \SpecialCharTok{/} \DecValTok{5}\NormalTok{)}
\NormalTok{x\_names[selected\_vars]}
\CommentTok{\# [1] "lab1\_actb"             "lab1\_molecules\_score"  "lab1\_telomerase\_score"}
\end{Highlighting}
\end{Shaded}

\hypertarget{intrinsic-selection-with-missing-data}{%
\subsubsection{Intrinsic selection with missing
data}\label{intrinsic-selection-with-missing-data}}

To perform intrinsic variable selection using the imputed data, we first
estimate variable importance for each imputed dataset.

\begin{Shaded}
\begin{Highlighting}[]
\FunctionTok{set.seed}\NormalTok{(}\DecValTok{20231121}\NormalTok{)}
\NormalTok{est\_lst }\OtherTok{\textless{}{-}} \FunctionTok{lapply}\NormalTok{(}\FunctionTok{as.list}\NormalTok{(}\DecValTok{1}\SpecialCharTok{:}\DecValTok{5}\NormalTok{), }\ControlFlowTok{function}\NormalTok{(l) \{}
\NormalTok{  this\_x }\OtherTok{\textless{}{-}}\NormalTok{ imputed\_biomarkers }\SpecialCharTok{\%\textgreater{}\%}
    \FunctionTok{filter}\NormalTok{(imp }\SpecialCharTok{==}\NormalTok{ l) }\SpecialCharTok{\%\textgreater{}\%}
    \FunctionTok{select}\NormalTok{(}\FunctionTok{starts\_with}\NormalTok{(}\StringTok{"lab"}\NormalTok{), }\FunctionTok{starts\_with}\NormalTok{(}\StringTok{"cea"}\NormalTok{))}
\NormalTok{  this\_y }\OtherTok{\textless{}{-}}\NormalTok{ biomarkers}\SpecialCharTok{$}\NormalTok{mucinous}
  \FunctionTok{suppressWarnings}\NormalTok{(}
    \FunctionTok{sp\_vim}\NormalTok{(}\AttributeTok{Y =}\NormalTok{ this\_y, }\AttributeTok{X =}\NormalTok{ this\_x, }\AttributeTok{V =}\NormalTok{ V, }\AttributeTok{type =} \StringTok{"auc"}\NormalTok{, }
    \AttributeTok{SL.library =}\NormalTok{ learners, }\AttributeTok{gamma =} \FloatTok{0.1}\NormalTok{, }\AttributeTok{alpha =} \FloatTok{0.05}\NormalTok{, }\AttributeTok{delta =} \DecValTok{0}\NormalTok{,}
    \AttributeTok{cvControl =} \FunctionTok{list}\NormalTok{(}\AttributeTok{V =}\NormalTok{ V), }\AttributeTok{env =} \FunctionTok{environment}\NormalTok{())}
\NormalTok{  )}
\NormalTok{\})}
\end{Highlighting}
\end{Shaded}

Next, we use Rubin's rules {[}27{]} to combine the variable importance
estimates, and use this to perform variable selection. Here, we control
the probability of making more than 5 errors (the generalized
family-wise error rate) at 5\%, using Holm-adjusted p-values to control
the initial family-wise error rate.

\begin{Shaded}
\begin{Highlighting}[]
\NormalTok{intrinsic\_set }\OtherTok{\textless{}{-}} \FunctionTok{intrinsic\_selection}\NormalTok{(}
  \AttributeTok{spvim\_ests =}\NormalTok{ est\_lst, }\AttributeTok{sample\_size =} \FunctionTok{nrow}\NormalTok{(biomarkers),}
  \AttributeTok{feature\_names =}\NormalTok{ x\_names, }\AttributeTok{alpha =} \FloatTok{0.05}\NormalTok{, }
  \AttributeTok{control =} \FunctionTok{list}\NormalTok{(}\AttributeTok{quantity =} \StringTok{"gFWER"}\NormalTok{, }\AttributeTok{base\_method =} \StringTok{"Holm"}\NormalTok{, }\AttributeTok{k =} \DecValTok{5}\NormalTok{)}
\NormalTok{)}
\FunctionTok{library}\NormalTok{(}\StringTok{"dplyr"}\NormalTok{)}
\NormalTok{intrinsic\_set }\SpecialCharTok{\%\textgreater{}\%}
  \FunctionTok{filter}\NormalTok{(selected)}
\CommentTok{\# A tibble: 5 × 6}
\CommentTok{\#   feature                            est p\_value adjusted\_p\_value  rank selected}
\CommentTok{\#   \textless{}chr\textgreater{}                            \textless{}dbl\textgreater{}   \textless{}dbl\textgreater{}            \textless{}dbl\textgreater{} \textless{}dbl\textgreater{} \textless{}lgl\textgreater{}   }
\CommentTok{\# 1 lab1\_actb                       0.0334   0.409                1     2 TRUE    }
\CommentTok{\# 2 lab1\_telomerase\_score           0.0216   0.439                1     5 TRUE    }
\CommentTok{\# 3 lab3\_muc3ac\_score               0.0238   0.433                1     4 TRUE    }
\CommentTok{\# 4 lab6\_ab\_score                   0.0241   0.433                1     3 TRUE    }
\CommentTok{\# 5 lab2\_fluorescence\_mucinous\_call 0.0354   0.396                1     1 TRUE}
\end{Highlighting}
\end{Shaded}

We select five variables, here those with the top-5 estimated variable
importance. The point estimates and p-values have been computed using
Rubin's rules. Two of the variables are the same as those selected using
extrinsic selection.

\hypertarget{conclusion}{%
\section{Conclusion}\label{conclusion}}

add a conclusion here

\hypertarget{availability}{%
\section{Availability}\label{availability}}

The \texttt{flevr} package is publicly available on
\href{https://github.com/bdwilliamson/flevr}{GitHub}; stable releases
are publicly available on the
\href{https://cran.r-project.org/package=flevr}{Comprehensive R Archive
Network}. Documentation and examples may be found in the package manual
pages and vignettes, and on the \texttt{pkgdown} website {[}28{]} at
\url{https://bdwilliamson.github.io/flevr}.

\hypertarget{acknowledgements}{%
\section{Acknowledgements}\label{acknowledgements}}

The authors gratefully acknowledge bug reports and feature requests
submitted by Bhavesh Borate.

\hypertarget{refs}{}
\begin{CSLReferences}{0}{0}
\leavevmode\vadjust pre{\hypertarget{ref-leng2006}{}}%
\CSLLeftMargin{1. }%
\CSLRightInline{Leng C, Lin Y, Wahba G. A note on the lasso and related
procedures in model selection. Statistica Sinica. 2006;16: 1273--1284.
Available: \url{https://www.jstor.org/stable/24307787}}

\leavevmode\vadjust pre{\hypertarget{ref-heymans2007}{}}%
\CSLLeftMargin{2. }%
\CSLRightInline{Heymans MW, Buuren S van, Knol DL, Van Mechelen W, De
Vet HC. Variable selection under multiple imputation using the bootstrap
in a prognostic study. BMC Medical Research Methodology. 2007;7: 1--10.
doi:\href{https://doi.org/10.1186/1471-2288-7-33}{10.1186/1471-2288-7-33}}

\leavevmode\vadjust pre{\hypertarget{ref-little1985}{}}%
\CSLLeftMargin{3. }%
\CSLRightInline{Little R, Schluchter M. Maximum likelihood estimation
for mixed continuous and categorical data with missing values.
Biometrika. 1985;72: 497--512. }

\leavevmode\vadjust pre{\hypertarget{ref-tsiatis2007}{}}%
\CSLLeftMargin{4. }%
\CSLRightInline{Tsiatis A. Semiparametric theory and missing data.
Springer Science \& Business Media; 2007. }

\leavevmode\vadjust pre{\hypertarget{ref-bang2005}{}}%
\CSLLeftMargin{5. }%
\CSLRightInline{Bang H, Robins J. Doubly robust estimation in missing
data and causal inference models. Biometrics. 2005;61: 962--973. }

\leavevmode\vadjust pre{\hypertarget{ref-johnson2008}{}}%
\CSLLeftMargin{6. }%
\CSLRightInline{Johnson B, Lin D, Zeng D. Penalized estimating functions
and variable selection in semiparametric regression models. Journal of
the American Statistical Association. 2008;103: 672--680. }

\leavevmode\vadjust pre{\hypertarget{ref-wolfson2012}{}}%
\CSLLeftMargin{7. }%
\CSLRightInline{Wolfson J. {EEB}oost: A general method for prediction
and variable selection based on estimating equations. Journal of the
American Statistical Association. 2011;106. }

\leavevmode\vadjust pre{\hypertarget{ref-rubin1987}{}}%
\CSLLeftMargin{8. }%
\CSLRightInline{Rubin D. Multiple imputation for nonresponse in surveys.
John Wiley \& Sons; 1987. }

\leavevmode\vadjust pre{\hypertarget{ref-vanbuuren2007}{}}%
\CSLLeftMargin{9. }%
\CSLRightInline{Buuren S van. Multiple imputation of discrete and
continuous data by fully conditional specification. Statistical Methods
in Medical Research. 2007;16: 219--242. }

\leavevmode\vadjust pre{\hypertarget{ref-mice2011}{}}%
\CSLLeftMargin{10. }%
\CSLRightInline{van Buuren S, Groothuis-Oudshoorn K. {mice}:
Multivariate imputation by chained equations in r. Journal of
Statistical Software. 2011;45: 1--67.
doi:\href{https://doi.org/10.18637/jss.v045.i03}{10.18637/jss.v045.i03}}

\leavevmode\vadjust pre{\hypertarget{ref-tibshirani1996}{}}%
\CSLLeftMargin{11. }%
\CSLRightInline{Tibshirani R. Regression shrinkage and selection via the
lasso. Journal of the Royal Statistical Society: Series B (Statistical
Methodology). 1996; 267--288. }

\leavevmode\vadjust pre{\hypertarget{ref-glmnet}{}}%
\CSLLeftMargin{12. }%
\CSLRightInline{Friedman J, Hastie T, Tibshirani R, Narasimhan B, Tay K,
Simon N, et al. {glmnet}: Lasso and elastic-net regularized generalized
linear models. 2023. Available:
\url{https://CRAN.R-project.org/package=glmnet}}

\leavevmode\vadjust pre{\hypertarget{ref-meinshausen2010}{}}%
\CSLLeftMargin{13. }%
\CSLRightInline{Meinshausen N, Bühlmann P. Stability selection. Journal
of the Royal Statistical Society: Series B (Statistical Methodology).
2010;72: 417--473. }

\leavevmode\vadjust pre{\hypertarget{ref-Hofner:StabSel:2015}{}}%
\CSLLeftMargin{14. }%
\CSLRightInline{Hofner B, Boccuto L, Göker M. Controlling false
discoveries in high-dimensional situations: Boosting with stability
selection. {BMC Bioinformatics}. 2015;16: 144. Available:
\url{http://dx.doi.org/10.1186/s12859-015-0575-3}}

\leavevmode\vadjust pre{\hypertarget{ref-barber2015}{}}%
\CSLLeftMargin{15. }%
\CSLRightInline{Barber R, Candès E. Controlling the false discovery rate
via knockoffs. Annals of Statistics. 2015;43: 2055--2085. }

\leavevmode\vadjust pre{\hypertarget{ref-knockoff}{}}%
\CSLLeftMargin{16. }%
\CSLRightInline{Patterson E, Sesia M. Knockoff: The knockoff filter for
controlled variable selection. 2022. Available:
\url{https://CRAN.R-project.org/package=knockoff}}

\leavevmode\vadjust pre{\hypertarget{ref-miselect}{}}%
\CSLLeftMargin{17. }%
\CSLRightInline{Rix A, Du J. {miselect}: Variable selection for multiply
imputed data. 2020. Available:
\url{https://cran.r-project.org/package=miselect}}

\leavevmode\vadjust pre{\hypertarget{ref-du2022variable}{}}%
\CSLLeftMargin{18. }%
\CSLRightInline{Du J, Boss J, Han P, Beesley LJ, Kleinsasser M, Goutman
SA, et al. Variable selection with multiply-imputed datasets: Choosing
between stacked and grouped methods. Journal of Computational and
Graphical Statistics. 2022;31: 1063--1075. }

\leavevmode\vadjust pre{\hypertarget{ref-breiman2001}{}}%
\CSLLeftMargin{19. }%
\CSLRightInline{Breiman L. Random forests. Machine Learning. 2001;45:
5--32. }

\leavevmode\vadjust pre{\hypertarget{ref-xin2012}{}}%
\CSLLeftMargin{20. }%
\CSLRightInline{Xin L, Zhu M. Stochastic stepwise ensembles for variable
selection. Journal of Computational and Graphical Statistics. 2012;21:
275--294. }

\leavevmode\vadjust pre{\hypertarget{ref-saeys2008}{}}%
\CSLLeftMargin{21. }%
\CSLRightInline{Saeys Y, Abeel T, Van de Peer Y. Robust feature
selection using ensemble feature selection techniques. Joint european
conference on machine learning and knowledge discovery in databases.
2008. pp. 313--325. }

\leavevmode\vadjust pre{\hypertarget{ref-r_software}{}}%
\CSLLeftMargin{22. }%
\CSLRightInline{R Core Team. R: A language and environment for
statistical computing. Vienna, Austria: R Foundation for Statistical
Computing; 2023. Available: \url{https://www.R-project.org/}}

\leavevmode\vadjust pre{\hypertarget{ref-vanderlaan2007super}{}}%
\CSLLeftMargin{23. }%
\CSLRightInline{Laan M van der, Polley E, Hubbard A. Super learner.
Statistical Applications in Genetics and Molecular Biology. 2007;6:
Online Article 25.
doi:\href{https://doi.org/10.2202/1544-6115.1309}{10.2202/1544-6115.1309}}

\leavevmode\vadjust pre{\hypertarget{ref-williamson2021vim2}{}}%
\CSLLeftMargin{24. }%
\CSLRightInline{Williamson B, Gilbert P, Simon N, Carone M. A general
framework for inference on algorithm-agnostic variable importance.
Journal of the American Statistical Association (Theory \& Methods).
2021.
doi:\href{https://doi.org/10.1080/01621459.2021.2003200}{10.1080/01621459.2021.2003200}}

\leavevmode\vadjust pre{\hypertarget{ref-williamson2020spvim}{}}%
\CSLLeftMargin{25. }%
\CSLRightInline{Williamson B, Feng J. Efficient nonparametric
statistical inference on population feature importance using {S}hapley
values. Proceedings of the 37th international conference on machine
learning. 2020. pp. 10282--10291. Available:
\url{http://proceedings.mlr.press/v119/williamson20a.html}}

\leavevmode\vadjust pre{\hypertarget{ref-williamson2023flevr}{}}%
\CSLLeftMargin{26. }%
\CSLRightInline{Williamson B, Huang Y. Flexible variable selection in
the presence of missing data. International Journal of Biostatistics.
2023. Available: \url{https://arxiv.org/abs/2202.12989}}

\leavevmode\vadjust pre{\hypertarget{ref-rubin2018}{}}%
\CSLLeftMargin{27. }%
\CSLRightInline{Rubin DB. Multiple imputation. Flexible imputation of
missing data, second edition. Chapman; Hall/CRC; 2018. pp. 29--62.
doi:\href{https://doi.org/10.1201/9780429492259}{10.1201/9780429492259}}

\leavevmode\vadjust pre{\hypertarget{ref-wickham_pkgdown}{}}%
\CSLLeftMargin{28. }%
\CSLRightInline{Wickham H, Hesselberth J. {pkgdown}: Make static HTML
documentation for a package. 2022. Available:
\url{https://CRAN.R-project.org/package=pkgdown}}

\end{CSLReferences}

\nolinenumbers



\end{document}
